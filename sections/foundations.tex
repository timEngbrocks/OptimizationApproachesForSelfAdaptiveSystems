\section{Foundations}
\label{ch:Foundations}

\noindent In essence \acrshort{sas} are systems which change themselves and therefore their own behavior
according to predefined rules and policies when they detect that predefined conditions are met or states have been reached.

%% TODO: Foundations
\noindent To better understand \acrshort{sas}, let us take a look at two examples.

\subparagraph*{Example 1}
Imagine you are managing an online store.
Your store offers multiple services.
Customers can browse your selection and buy your products.
In addition to that customers can select products for which they want to receive a notification when they drop to a certain price.
Business customers have the option of receiving their bill via Email.
Lastly you manage your own advertisements.

\noindent Based on these services there are some system parameters that you can control.
You can change which products and advertisements are served to a customer.
Based on how customers set price alerts you can change the price of a product to potentially increase sales.
Lastly you can change how often your system sends Emails to decrease the load of your Email server.

\noindent While you could perform all these tasks on your own, it would be more efficient to automate them.
For this purpose you choose to implement a \acrshort{sas}.
This \acrshort{sas} automatically adapts the price of products based on sales performance metrics that you define.
It also chooses which products and advertisements to serve to a customer based on their purchase history
and marketing policies that you create.
Additionally the \acrshort{sas} controls the frequency of sending Emails by monitoring the load of the Email server
and a goal that you selected.

\noindent In summary, this \acrshort{sas} monitors itself and its environment
and then adapts itself according to rules and policies that you defined.

\subparagraph*{Example 2}
Imagine you are developing the software for a robot.
The robot should be able to navigate through partially known terrain,
collect data based on some metrics and be able to update its software when a new version is released.

\noindent To navigate through terrain that is only partially known, the robot has to be able to adapt
its movement to an environment which it has not experienced before. 
To collect data based on some performance metrics, the robot needs to change its method
of data collection when the selected metrics change.
Lastly, to update its software autonomously, the robots needs to be able
to modify its software and reconfigure itself.

\noindent All these tasks can be implemented as a \acrshort{sas}.
Firstly, the robot might use a system that selects
which strategy should be used for navigation based on images of its environment.
You could then define a set of different strategies for the robot's navigation.
For example, one strategy for steep terrain and one for terrain with an even ground.

\noindent For the data collection you could define different policies
that decide which data the robot should collect if it detects that a metric has reached a threshold.
An example could be that if the robot is certain that it detected a human with its camera,
that it should not store images of that person to avoid privacy conflicts.

\noindent The self-updating behavior only requires the monitoring of one environment attribute:
If a new version of the robot's software has been released.
When this is the case, you define a policy that the robot should stop moving, download the new software
and adapt itself.

\noindent These two examples illustrate how different systems can use self-adaptiveness.
In most cases it is enough to define simple rules and policies that should be executed
when some condition is met.

% \noindent To better understand \acrshort{sas}, let us take a look at a commonly used example.
% Imaging you are the system administrator of a large scale online store.
% This store has four parameters: site traffic, number of purchases, number of active server instances and the number of served advertisements.
% During your day to day operations you encounter a common type of task:
% Update some system parameter X based on some metric Y.
% To make your job easier, you decide to use a \acrshort{sas} for these tasks
% and come up with the following generalized adaptation rule:
% If metric Y crosses threshold Z, update the system parameter X.

% \noindent In this case the usage of a \acrshort{sas} would be beneficial because it could easily
% automate a general set of tasks.
% A human operator might have been able to perform these tasks on his own,
% if the number of system parameters was sufficiently small.
% But the \acrshort{sas} is better at handling large numbers of system parameters.

% \noindent As explained in the previous chapter,
% the performance of a \acrshort{sas} degrades over time
% as the expected results of adaptation diverge from the actual adaptation results
% because of unforeseen changes in the environment.

% \noindent This can be illustrated by the previous example.
% Imagine that the \acrshort{sas} has been in operation for some and you collected data on the systems performance.
% You notice that the \acrshort{sas} changes some system parameters too aggressively,
% because it performs an adaptation as soon as a metrics threshold has been violated.
% As a human operator you would have waited some time to see how the metric develops before performing an adaptation
% which would result in a smoother operation.
% The \acrshort{sas} can not handle this type of uncertainty and only reacted to the current state of its environment.



% \noindent The previous example could benefit from such optimizations by dynamically updating adaptation rules
% to better reflect the systems changing environment.
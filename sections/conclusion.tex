\newpage
\section{Conclusion}
\label{ch:Conclusion}

While there are different approaches to classify and reason about \acrshort{sas}, three of which were highlighted by this paper,
there is no classification for \acrshort{oa} for \acrshort{sas}.
Because of this it is hard to compare different \acrshort{oa} and identify areas which require further research.
To solve this problem this paper focused on proposing a classification for \acrshort{oa} for \acrshort{sas}.

\noindent To accomplish this, chapter \ref{ch:Foundations} started by explaining how \acrshort{sas} work with two short examples
and also showed some of their limitations.
This was followed by chapter \ref{ch:SASClassification} which explored three different approaches
for classifying and reasoning about \acrshort{sas}.
Chapter \ref{ch:Proposal} then derived and proposed a classification for \acrshort{oa} based on the principles shown in the previous chapter.
Lastly, chapter \ref{ch:Existing} compared a selection of existing \acrshort{oa} by using the proposed classification.
The comparison was then used to draw some initial conclusions about \acrshort{oa}.

\noindent The first conclusion was that most \acrshort{oa} only use the systems runtime to perform optimizations.
This has the advantage of not needing to simulate the system and its environment which can be complex.
But because most \acrshort{oa} use machine learning for their optimizations,
this can lead to unexpected or unwanted behavior at the beginning of the systems runtime.
This disadvantage is especially present in \acrshort{oa} that use machine learning approaches
like Q-learning which actively explores the problem space to find optimal solutions.

\noindent The second conclusion was that only a few \acrshort{oa} search for problems which require optimization in areas of a \acrshort{sas}
besides the Adaptation Control. The other two areas that can be optimized: the Level and the Technique still require
research into how effective their optimization can be.

\noindent The last conclusion was that most \acrshort{oa} use centralized approaches for performing optimizations.
Because of this the usage of decentralized approaches still requires further research.

\noindent Even though the proposed classification helps with comparing the existing \acrshort{oa},
it does not help with comparing their performance.
This is a problem not only for \acrshort{oa} but for \acrshort{sas} as well.
There is currently no scientific way of comparing the performance of \acrshort{sas}.
Because of this the effect of \acrshort{oa} on \acrshort{sas} can also not be quantified.

\noindent Another problem that \acrshort{sas} and their \acrshort{oa} have is that
most of them are highly domain specific and only try to solve a single domain problem.
This means that domain independent \acrshort{sas} and \acrshort{oa} for them still require research.
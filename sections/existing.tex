\newpage
\section{Classifying existing optimization approaches}
\label{ch:Existing}

% TODO:
% - Write a short explanation of each approach:
%     - FUSION
%     - Multi-Agent system
%     - Learning revised models
%     - Cloud performance and security
%     - FIoT
% - Classify each approach:
%     - FUSION
%     - Multi-Agent system
%     - Learning revised models
%     - Cloud performance and security
%     - FIoT

% \begin{itemize}
%     \item Classifying a selection of existing optimization approaches using the proposed classification.
% \end{itemize}

% \paragraph*{Literature for this section:} \begin{itemize}
%     \item "FUSION: a framework for engineering self-tuning self-adaptive software systems" \cite{FUSION}
%     \item "FIoT: An agent-based framework for self-adaptive and self-organizing applications based on the Internet of Things" \cite{FIoT}
%     \item "A multi-agent systems approach to autonomic computing" \cite{MultiAgentSystem}
%     \item "Learning revised models for planning in adaptive systems" \cite{LearningRevisedModels}
%     \item "Using a multi-agent system and artificial intelligence for monitoring and improving the cloud performance and security" \cite{ImprovingCloudPerformanceAndSecurity}
%     \item FESAS by Krupitzer? TODO: research
% \end{itemize}

\subparagraph*{FUSION}
by Elkhodary et al, 2010 \cite*{FUSION}  uses a learning and an adaptation cycle.
The adaptation cycle corresponds to the Self-Adaptive part of the system
and does not directly affect the learning cycle.
The learning cycle is the optimization approach used by FUSION,
it changes the behavior of the adaptation cycle.
Both of these cycles continuously run in parallel to each other.
\begin{itemize}
    \item Location: Adaptation Control + Technique ?
    \item Time: FUSION uses both the design and run time to perform optimizations.
    The design time is used to generate an initial model for the learning cycle.
    During the run time the learning cycle monitors adaptations
    and updates the Knowledge Base to improve future adaptations.
    \item Technique:
    \item Purpose: The goal of FUSION is to 
    \item Approach: FUSION acts as an external actor.
    \item Reflective Adaptation Control: 
\end{itemize}

\subparagraph*{FIoT}
by 
\begin{itemize}
    \item Location:
    \item Time:
    \item Technique:
    \item Purpose:
    \item Approach:
    \item Reflective Adaptation Control:
\end{itemize}

\subparagraph*{FUSION}
\begin{itemize}
    \item Location:
    \item Time:
    \item Technique:
    \item Purpose:
    \item Approach:
    \item Reflective Adaptation Control:
\end{itemize}

\subparagraph*{FUSION}
\begin{itemize}
    \item Location:
    \item Time:
    \item Technique:
    \item Purpose:
    \item Approach:
    \item Reflective Adaptation Control:
\end{itemize}

\subparagraph*{FUSION}
\begin{itemize}
    \item Location:
    \item Time:
    \item Technique:
    \item Purpose:
    \item Approach:
    \item Reflective Adaptation Control:
\end{itemize}
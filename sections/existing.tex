\section{Classifying Existing Optimization Approaches}
\label{ch:Existing}

% TODO:
% - compare approaches
% - one more approach
% \paragraph*{Literature for this section:} \begin{itemize}
%     \item "A multi-agent systems approach to autonomic computing" \cite{MultiAgentSystem}
%     \item "Using a multi-agent system and artificial intelligence for monitoring and improving the cloud performance and security" \cite{ImprovingCloudPerformanceAndSecurity}
% \end{itemize}

\subparagraph*{FUSION}
by Elkhodary et al, 2010 \cite*{FUSION}  uses a learning and an adaptation cycle.
The adaptation cycle corresponds to the Self-Adaptive part of the system
and does not directly affect the learning cycle.
The learning cycle is the \acrlong{oa} used by FUSION,
it changes the behavior of the adaptation cycle.
Both of these cycles continuously run in parallel to each other.
\begin{itemize}[nosep]
    \item Location: FUSION optimizes the Adaptation Control.
    \item Time: FUSION uses both the design and run time to perform optimizations.
    The design time is used to generate an initial model for the learning cycle.
    During the run time the learning cycle monitors adaptations
    and updates the Knowledge Base to improve future adaptations.
    \item Technique: FUSION updates the rules used by the Self-Adaptive System and the shared knowledge base.
    \item Purpose: FUSION aims to decrease the difference between the expected and actual outcome of adaptation rules.
    \item Approach: FUSION acts as an external actor.
    \item Reflective Adaptation Control: FUSION applies optimizations by updating models and modifying rules.
\end{itemize}

\subparagraph*{Learning Revised Models for Planning in Adaptive Systems}
by Sykes et al., 2013 \cite*{LearningRevisedModels} uses
"a non-monotonic
probabilistic rule learning technique, NoMPRoL, that finds
hypotheses consisting of new rules specifying the probability
of an action achieving its specified outcome under particular
conditions" (Sykes et al., 2013 \cite*{LearningRevisedModels})
\begin{itemize}[nosep]
    \item Location: The optimizations focus on the Adaptation Control.
    \item Time: The run time is used to perform optimizations.
    \item Technique: This approach optimizes the adaptation rules and domain models.
    \item Purpose: The optimizations aim to generate new and better adaptation rules and domain models.
    \item Approach: The optimizations are performed by an internal component that sits between the domain model
    and the Self-Adaptive System.
    \item Reflective Adaptation Control: This approach generates new adaptation rules.
\end{itemize}

\subparagraph*{FIoT}
by Moraes do Nascimento and Pereira, 2016 \cite*{FIoT} is a framework for using agent-based systems
to construct self-organizing systems for the Internet of Things.
\begin{itemize}[nosep]
    \item Location: FIoT optimizes three different levels: The physical, communication and application layer.
    \item Time: FIoT performs its optimizations during the run time of the system.
    \item Technique: FIoT changes the Level, Adaptation Technique and Rules / Policies of a Self-Adaptive System.
    \item Purpose: Generate self-organizing behavior in a Self-Adaptive System.
    \item Approach: FIoT uses a decentralized agent-based approach.
    \item Reflective Adaptation Control: % TODO:
\end{itemize}
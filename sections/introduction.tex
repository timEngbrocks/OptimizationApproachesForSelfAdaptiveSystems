\newpage
\section{Introduction}
\label{ch:Introduction}

% TODO:
% - Today large systems ....
% - Need for SAS
% - What are the limits
% - The need for optimization that results from those limits
% - The need to classify these approaches

% \begin{itemize}
%     \item What are Self-Adaptive Systems?
%     \item Why are Self-Adaptive Systems useful?
%     \item What are the limits of classical Self-Adaptive Systems?
%     \item The need for optimizing Self-Adaptive Systems.
% \end{itemize}

% \paragraph*{Literature for this section:} \begin{itemize}
%     \item "The vision of autonomic computing" \cite{VisionOfAutonomicComputing}
%     \item "An Introduction to Self-adaptive Systems: A Contemporary Software Engineering Perspective" \cite{SasIntroduction}
%     \item "Software Engineering for Self-Adaptive Systems: A Research Roadmap" \cite{ResearchRoadmap}
%     \item "Software Engineering for Self-Adaptive Systems: A Second Research Roadmap" \cite{SecondResearchRoadmap}
%     \item "Claims and supporting evidence for self-adaptive systems: A literature study" \cite{ClaimsAndSupportingEvidence}
%     \item "Self-Adaptive Software: Landscape and Research Challenges" \cite{LandscapeAndResearchChallenges}
% \end{itemize}

% TODO: What are Self-Adaptive Systems?}
% TODO: Why are Self-Adaptive Systems useful?}
\par
The complexity of modern software systems is constantly growing.
Most of this growth in complexity stems from the 
"need to integrate several heterogeneous environments into corporate-wide computing systems, 
and to extend that beyond company boundaries into the Internet" (Kephart and Chess, 2003 \cite*{VisionOfAutonomicComputing}).
\newline
This has reached a state where the 
"complexity appears to be approaching the limits of human capability" (Kephart and Chess, 2003 \cite*{VisionOfAutonomicComputing}).
In combination with the uncertainty about a software systems future operations and environment,
that the developers of such complex systems face, it becomes uneconomical to purely operate a system by human operators.
\newline
\par


From this the need for software systems which can autonomously manage themselves arises.
In order to achieve this task of autonomous self-management, the system has to be able to:
% MAPE-K -> Monitor, Analyze, Plan, Execute with Knowledge
\begin{itemize}
    \item detect faults and changes in its environment.
    \item decide how to react to faults and changes in the systems environment.
    \item make changes to itself.
\end{itemize}
To model these abilities Kephart and Chess developed 
the MAPE-K (Monitor-Analyze-Plan-Execute with Knowledge) feedback loop \cite*{VisionOfAutonomicComputing}.
\newline
First the system has to monitor itself and its environment. 
The data, gathered by the monitoring step, has to be analyzed to detect changes and faults.
If the analyzing step detects, that an adaptation is necessary, 
the planning step plans which changes have to be made.
After the changes have been planned, they need to be executed.
All of this happens with Knowledge of the environment and the system.
\newline
Software systems that can autonomously manage themselves are called Self-Adaptive Systems
because of their ability to adapt themselves.
\newline
\par


This approach to managing software systems has advantages compared to the use of human operators:
\begin{itemize}% TODO: list advantages.
    \item 
\end{itemize}

% TODO: What are the limits of Self-Adaptive Systems?}
% TODO: The need for optimizing Self-Adaptive Systems}
While Self-Adaptive Systems are better at handling more complex systems,
human operators are better at handling uncertainty.
\newline
This is because the rules and policies, used by Self-Adaptive Systems,
are created at design time.
The consequence of this is, that Self-Adaptive Systems can adapt themselves under a changing environment,
but they can not adapt the rules and policies that are used for adaptation.
In combination with the fact that not all environment changes can be predicted during design time,


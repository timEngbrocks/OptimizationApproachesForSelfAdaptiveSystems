%% LaTeX2e class for seminar theses
%% seminar.tex
%% 
%% Karlsruhe Institute of Technology
%% Institute for Program Structures and Data Organization
%% Chair for Software Design and Quality (SDQ)
%%
%% Dr.-Ing. Erik Burger
%% burger@kit.edu
%%
%% Version 1.0.3, 2020-06-26

%% Available page modes: oneside, twoside
%% Available languages: english, ngerman
%% Available modes: draft, final (see README)
\documentclass[twoside, english]{sdqseminar}

%% ---------------------------------
%% | Information about the thesis  |
%% ---------------------------------

%% Name of the author
\author{Tim Engbrocks}

%% Title (and possibly subtitle) of the thesis
\title{Optimization Approaches for Self-Adaptive Systems}

%% Type of the thesis 
% \thesistype{Seminar Thesis}

%% Change the institute here, ``IPD'' is default
% \myinstitute{Institute for \dots}

%% The advisors are PhD Students or Postdocs
\advisor{Dipl.-Inform. Martina Rapp}

\settitle

%% --------------------------------
%% | Settings for word separation |
%% --------------------------------

%% Describe separation hints here.
%% For more details, see 
%% http://en.wikibooks.org/wiki/LaTeX/Text_Formatting#Hyphenation
\hyphenation{
% me-ta-mo-del
}

%% ----------------------------
%% | Acronyms                 |
%% ----------------------------
\usepackage[acronym]{glossaries}
\loadglsentries[type=\acronymtype]{sections/glossary.tex}

%% --------------------------------
%% | Bibliography                 |
%% --------------------------------

%% Use biber instead of BibTeX, see README
\usepackage[citestyle=numeric,style=numeric,backend=biber]{biblatex}
\addbibresource{seminar.bib}

%% ====================================
%% ====================================
%% ||                                ||
%% || Beginning of the main document ||
%% ||                                ||
%% ====================================
%% ====================================
\begin{document}

%% Set PDF metadata
\setpdf

%% Set the title
\maketitle

%% ----------------
%% |   Abstract   |
%% ----------------

% Peer Review TODO:
% - Introduction
%   - Define terms
%   - Explain basic concepts better
% - Provide real world example for better understandability
% - Better explain FORMS
%   - Z notation
%   - What does reflective mean in this context
% - Mapping OA and MAPE-K redundant
% - Use short hand for \acrlong{sas}[s]: SAS
% - Groß-/Kleinschreibung von gewissen Begriffen
% - Existing FIoT: RAC fehlt noch
% - Komma Anmerkungen aus Peer Review #2
% - Bei Exisiting müssen die Approaches noch verglichen werden

% TODO:
% - Terms aus Diagrammen und Modellen kursiv schreiben
% - Acronyms überall verwenden
% - Figure placement
% - Text styling: paragraphs etc.

\begin{abstract}
    Content of the abstract: \begin{itemize}
        \item Motivating Self-Adaptive Systems and their need for optimization.
        \item Motivating the need for a classification of optimization approaches for Self-Adaptive Systems.
        \item The scope and goal of this paper.
    \end{itemize}
\end{abstract}

%% LaTeX2e class for seminar theses: Declaration of independent work
%% sections/declaration.tex
%% 
%% Karlsruhe Institute of Technology
%% Institute for Program Structures and Data Organization
%% Chair for Software Design and Quality (SDQ)
%%
%% Dr.-Ing. Erik Burger
%% burger@kit.edu
%%
%% Version 1.0.2, 2020-05-07

\thispagestyle{empty}
\null\vfill
\noindent\hbox to \textwidth{\hrulefill} 
\iflanguage{english}{I declare that I have developed and written the enclosed
thesis completely by myself, and have not used sources or means without
declaration in the text.}%
{Ich versichere wahrheitsgemäß, die Arbeit
selbstständig angefertigt, alle benutzten Hilfsmittel vollständig und genau
angegeben und alles kenntlich gemacht zu haben, was aus Arbeiten anderer
unverändert oder mit Änderungen entnommen wurde.}
 
 
%% ---------------------------------------------
%% | Replace PLACE and DATE with actual values |
%% ---------------------------------------------
\textbf{PLACE, DATE}
\vspace{1.5cm}
 
\dotfill\hspace*{8.0cm}\\
\hspace*{2cm}(\theauthor) 
\cleardoublepage

\section{Introduction}
\label{ch:Introduction}

\begin{figure*}[hbt!]
    \centering
    \includegraphics[width=0.6\textwidth]{images/MAPEK.png}
    \caption{The \acrshort{mapek} (Monitor-Analyze-Plan-Execute with Knowledge) feedback loop by Kephart and Chess, 2003 \cite*{VisionOfAutonomicComputing}}
    \label{fig:MAPEK}
\end{figure*}

The complexity of modern software systems is constantly growing.
Most of this growth in complexity stems from the
"need to integrate several heterogeneous environments into corporate-wide computing systems,
and to extend that beyond company boundaries into the Internet" (Kephart and Chess, 2003 \cite*{VisionOfAutonomicComputing}).
This has reached a state where the
"complexity appears to be approaching the limits of human capability" (Kephart and Chess, 2003 \cite*{VisionOfAutonomicComputing}).
In combination with the uncertainty about a software systems future operations and environment,
that the developers of such complex systems face, it becomes uneconomical to purely operate such a system by human operators.

\noindent From this the need for software systems which can autonomously manage themselves arises.
In order to achieve this task of autonomous self-management, the system has to be able to:
\begin{itemize}[nosep]
    \item detect faults and changes in its environment,
    \item analyze them,
    \item decide how to react to them
    \item and make changes to itself.
\end{itemize}

\newpage
\noindent To model these abilities Kephart and Chess developed
the \acrfull{mapek} feedback loop \cite*{VisionOfAutonomicComputing} shown in Figure \ref{fig:MAPEK}.

\noindent First the system has to \textit{monitor} itself and its environment.
The data, gathered by the monitoring step, has to be \textit{analyzed} to detect changes and faults.
If the analyzing step detects, that an adaptation is necessary,
the system has to \textit{plan} how to perform the necessary changes.
After the changes have been planned, they need to be \textit{executed}.
All of this happens with \textit{knowledge} of the environment and the system.

\noindent Software systems that can autonomously manage themselves are called \acrfull{sas}[s]
because of their ability to adapt themselves.

\noindent To better understand \acrshort{sas}, let us take a look at a commonly used example.
Imaging you are the system administrator of a large scale online store.
This store has four parameters: site traffic, number of purchases, number of active server instances and the number of served advertisements.
During your day to day operations you encounter a common type of task:
Update some system parameter X based on some metric Y.
To make your job easier, you decide to use a \acrshort{sas} for these tasks
and come up with the following generalized adaptation rule:
If metric Y crosses threshold Z, update the system parameter X.

\noindent In this case the usage of a \acrshort{sas} was beneficial because it could easily
automate a general set of tasks.
A human operator might have been able to perform these tasks on his own,
if the number of system parameters was sufficiently small.
But the \acrshort{sas} is better at handling large numbers of system parameters.


\noindent While \acrshort{sas} are better at handling more complex systems,
human operators are better at handling uncertainty.
This has two reasons. 
Firstly, the adaptation rules and policies used by \acrshort{sas} are statically created at design time.
Secondly, \acrshort{sas} can adapt the software that they are managing but they can not change their adaptation process.
Over time this leads to an increasing divergence between the expected results of adaptations and the actual results,
when the environment changes in ways that were not predicted by developers during the design time of the system.

\noindent This can be illustrated by the previous example.
Imagine that the \acrshort{sas} has been in operation for some and you collected data on the systems performance.
You notice that the \acrshort{sas} changes some system parameters too aggressively,
because it performs an adaptation as soon as a metrics threshold has been violated.
As a human operator you would have waited some time to see how the metric develops before performing an adaptation
which would result in a smoother operation.
The \acrshort{sas} can not handle this type of uncertainty and only reacted to the current state of its environment.

\noindent Optimizations are necessary to improve the performance and effectiveness of \acrshort{sas} in situations like these.
There are already many approaches on how to optimize \acrshort{sas}.
Some of them focus on updating adaptation rules and policies during the systems runtime.
Others dynamically change at which level of the system adaptations are performed.
Generally most optimizations target static components of \acrshort{sas}.
These components can be improved by making them more dynamic, 
which is often achieved by applying modern learning methods.

\noindent The previous example could benefit from such optimizations by dynamically updating adaptation rules
to better reflect the systems changing environment.

\noindent While there are already many \acrfull{oa}[es] for \acrshort{sas},
there is no classification for them.
Because of this, the existing \acrshort{oa} can not be easily compared
and it is difficult to identify areas which require further research.
This paper aims to provide such a classification for \acrlong{oa}[es] for \acrlong{sas}[s].

\noindent To derive a classification for \acrshort{oa} for \acrshort{sas},
chapter \ref{ch:SASClassification} will first explain how \acrshort{sas} are classified
using three different approaches.
Based on these approaches, a classification for \acrshort{oa} for \acrshort{sas} will be derived
and proposed in chapter \ref{ch:Proposal}.
In chapter \ref{ch:Existing} the proposed classification will be applied to some existing \acrshort{oa}.
Lastly, chapter \ref{ch:Conclusion} will finish with a conclusion and recommendations for future research directions.

\section{Foundations}
\label{ch:Foundations}

\noindent In essence \acrshort{sas} are systems which change themselves and therefore their own behavior
according to predefined rules and policies when they detect that predefined conditions are met or states have been reached.

%% TODO: Foundations
\noindent To better understand \acrshort{sas}, let us take a look at two examples.

\subparagraph*{Example 1}
Imagine you are managing an online store.
Your store offers multiple services.
Customers can browse your selection and buy your products.
In addition to that customers can select products for which they want to receive a notification when they drop to a certain price.
Business customers have the option of receiving their bill via Email.
Lastly you manage your own advertisements.

\noindent Based on these services there are some system parameters that you can control.
You can change which products and advertisements are served to a customer.
Based on how customers set price alerts you can change the price of a product to potentially increase sales.
Lastly you can change how often your system sends Emails to decrease the load of your Email server.

\noindent While you could perform all these tasks on your own, it would be more efficient to automate them.
For this purpose you choose to implement a \acrshort{sas}.
This \acrshort{sas} automatically adapts the price of products based on sales performance metrics that you define.
It also chooses which products and advertisements to serve to a customer based on their purchase history
and marketing policies that you create.
Additionally the \acrshort{sas} controls the frequency of sending Emails by monitoring the load of the Email server
and a goal that you selected.

\noindent In summary, this \acrshort{sas} monitors itself and its environment
and then adapts itself according to rules and policies that you defined.

\subparagraph*{Example 2}
Imagine you are developing the software for a robot.
The robot should be able to navigate through partially known terrain,
collect data based on some metrics and be able to update its software when a new version is released.

\noindent To navigate through terrain that is only partially known, the robot has to be able to adapt
its movement to an environment which it has not experienced before. 
To collect data based on some performance metrics, the robot needs to change its method
of data collection when the selected metrics change.
Lastly, to update its software autonomously, the robots needs to be able
to modify its software and reconfigure itself.

\noindent All these tasks can be implemented as a \acrshort{sas}.
Firstly, the robot might use a system that selects
which strategy should be used for navigation based on images of its environment.
You could then define a set of different strategies for the robot's navigation.
For example, one strategy for steep terrain and one for terrain with an even ground.

\noindent For the data collection you could define different policies
that decide which data the robot should collect if it detects that a metric has reached a threshold.
An example could be that if the robot is certain that it detected a human with its camera,
that it should not store images of that person to avoid privacy conflicts.

\noindent The self-updating behavior only requires the monitoring of one environment attribute:
If a new version of the robot's software has been released.
When this is the case, you define a policy that the robot should stop moving, download the new software
and adapt itself.

\noindent These two examples illustrate how different systems can use self-adaptiveness.
In most cases it is enough to define simple rules and policies that should be executed
when some condition is met.
\newpage
\section{Classification of Self-Adaptive Systems}
\label{ch:SASClassification}

There are different approaches on how to classify and describe \acrshort{sas} which all focus on different usages.
The three approaches that will be highlighted by this section are:
\begin{itemize}[nosep]
    \item FORMS \cite*{FORMS}
    \item Berns' and Ghosh's definition of self-* properties \cite*{DissectingSelfProperties}
    \item Krupitzer's et al. taxonomy for \acrshort{sas} \cite*{SurveyOnEngineeringApproaches}
\end{itemize} 
These approaches contain ideas that will be used to construct the classification for
\acrshort{oa} for \acrshort{sas} in the next chapter.

\begin{figure*}[b!]
    \centering
    \includegraphics[width=\textwidth]{images/FORMS.png}
    \caption{FORMS primitives \cite*{FORMS}}
    \label{fig:FORMS}
\end{figure*}

\noindent With FORMS Weyns et al. propose a formal reference model for describing \acrshort{sas} \cite{FORMS}.
The goal of FORMS is to provide a well defined basis for talking and reasoning about \acrshort{sas}.
This is achieved by constructing \acrshort{sas} from a set of primitives which are defined in Z notation.
Z notation is used for formal specifications and is based upon Zermelo-Fraenkel set theory.
It is mathematically well-defined and often used to describe software systems.

\noindent Figure \ref{fig:FORMS} shows all the primitives used by FORMS and their relationships.
The \acrshort{sas} is split into different layers of subsystems.
Base-level refers to subsystems on the bottom most layer of the system.
Subsystems in layers above the base-level layer are called reflective.
Reflection refers to the ability of a system to inspect and change itself.

\noindent Each subsystem consists of two parts: its computation and its model.
Base-level subsystems have domain models which contain data about the environment.
In addition to that, reflective subsystems also have a reflection model which contains data about the system itself.
While base-level subsystems can use their computation to affect the environment,
reflective subsystems can affect the subsystems in the layer below themselves.

\noindent This approach can be understood with the example of a robot that uses motorized wheels
and some sensors to observe its environment.
The motor control software used by the robot would be placed in the base-level layers.
The sensors would be placed in the base-level layer as well.
These components are responsible for observing the environment and interacting with it.
On top of the base-level layer the robot might use a reflective layer which changes how the wheels are operated
based on the current weather conditions that it observes.
The top layer of the robots software could be an automatic updater 
which automatically updates the robots software to the latest version.
For this, the updater has to examine the robots software and modify it.
In other words, the updater is a reflective component.

\noindent The concept of different layers of reflection will be useful later to distinguish
\acrshort{oa} from \acrshort{sas}.

%% self-* properties
\begin{figure*}[t!]
    \includegraphics[width=\textwidth]{images/SelfProperties.png}
    \caption{self-* properties \cite*{DissectingSelfProperties}}
    \label{fig:SelfProperties}
\end{figure*}

\noindent Although some of the first papers on \acrshort{sas}, for example, 
"The Vision of Autonomic Computing" by Kephart and Chess \cite*{VisionOfAutonomicComputing} focused on self-adaptation,
there are other aspects of software systems that can benefit from the ideas introduced by \acrshort{sas}.
Berns and Ghosh identified and described such aspects which they call self-* properties \cite*{DissectingSelfProperties}.
They found the self-* properties that are depicted in Figure \ref{fig:SelfProperties}.

\noindent Berns and Ghosh describe \acrshort{sas} in terms of their interaction with their environment
and their reaction to external actions \cite*{DissectingSelfProperties}.
External actions are actions stemming from the systems environment which affect the system directly.
Safety predicates are used to describe goals of the system.
These external actions are called malicious if the intent of the action is the violation of a safety predicate.
A system also has a configuration which is the collection of its parameters.
Using these definitions, Berns and Ghosh describe their self-* properties in the following way:

\subparagraph*{Self-stabilization}
A system has the self-stabilization property if it can get from any starting configuration
to a configuration in which its safety predicates are fulfilled and stay in such a configuration afterwards.
In other words: The system can complete its starting procedure without requiring assistance.
\subparagraph*{Self-adapting}
A system is self-adapting (or managing) if it maintains, improves or restores safety predicates
without human intervention.
This definition matches the definition proposed by Kephart and Chess \cite*{VisionOfAutonomicComputing}
of systems which can autonomously manage themselves without human intervention.
\subparagraph*{Self-healing}
A system is self-healing if an external action can only cause a temporary violation of safety predicates.
This means that a self-healing system can recover from external actions on its own.
An example would be a system which can restart itself after a power outage.
\subparagraph*{Self-organizing}
A system is self-organizing if it maintains, improves or restores safety predicates after an external
action which involves processes joining or leaving the system. The recovery time per join or leave should be sublinear.
Peer-to-peer networks can be examples for self-organizing systems.
\subparagraph*{Self-protecting}
A system is self-protecting if it maintains one chosen safety predicate even when malicious external actions occur.
Meaning that a self-protecting system guarantees that it will never violate the chosen safety predicate.
A real world example for this would be a system which guarantees that it will not disclose personal data to unauthorized users.
\subparagraph*{Self-optimization}
When a system is self-optimizing, it can maximize or minimize the value of a utility function.
A system which starts and stops services based on demand to reduce energy usage is an example for a self-optimizing system.
\subparagraph*{Self-configuration}
A system possesses the self-configuration property if it can change its configuration to restore or improve a safety predicate.
\subparagraph*{Self-scaling}
A self-scaling system can maintain or improve a system property while an external action affects its scale.
This means that a system which regulates the number of instances of one of its services based, for example, on demand is self-scaling.
\subparagraph*{Self-immunity}
A system with self-immunity can restore safety predicates after the occurrence of external actions
and return to a state where no safety predicate is violated as if the external action never happened.
\subparagraph*{Self-containment}
A system is self-containing if a malicious external action can not compromise the whole system
and the system is able to eventually return to its normal operating state.

\noindent These self-* properties are useful to:
\begin{itemize}[nosep]
    \item Communicate the abilities and goals of a system.
    \item Establish well-defined goals for the system.
\end{itemize}
They show that there are aspects besides the management of software which can benefit from the ideas of self-adaptation.
This can be used as it distinguishes \acrshort{sas} from system with other self-* properties.
The only other self-* property that will be important for the rest of this paper is the self-optimization property.

\begin{figure*}[t!]
    \includegraphics[width=\textwidth]{images/KrupitzerTaxonomy.jpg}
    \caption{Taxonomy for \acrshort{sas} \cite*{SurveyOnEngineeringApproaches}}
    \label{fig:KrupitzerTaxonomy}
\end{figure*}

\noindent The taxonomy for \acrshort{sas} in Figure \ref{fig:KrupitzerTaxonomy}
is based upon the 5W+1H questions by Salehie and Tahvildari, 2009 \cite*{LandscapeAndResearchChallenges}.
These questions are: Where, When, What, Why, Who and How.
Each of these questions is responsible for a different aspect of \acrshort{sas} and corresponds to a dimension of the taxonomy.

\subparagraph*{Why}
First there needs to be a reason for a \acrshort{sas} to adapt. Why an adaptation should be performed is answered by the Reason dimension.
According to the taxonomy reasons for an adaptation can be changes in either the context, a technical resource or changes caused by the user.

\subparagraph*{Where}
The question of where asks on which level of the system changes need to occur.
The different levels on which changes can occur include:
\begin{itemize}[nosep]
    \item Different levels of applications from the operating system to a user application
    \item How systems communicate with each other
    \item The technical resources that are needed by the system
    \item The context in which the system operates
\end{itemize}

\subparagraph*{When}
While the original When-Question by Salehie and Tahvildari tries to understand all temporal aspects of \acrshort{sas},
including how frequently changes should occur and if they happen continously,
the taxonomy only answers the question when changes should be performed.
For this purpose the Time dimension differentiates between systems that perform changes proactively or reactively.
Reactive changes occur after, for example, a system metric has been violated.
Proactive changes occur before a system metric can be violated.

\subparagraph*{What}
In addition to the question of where and when changes should occur, it is also important to know
what changes should occur. There are different techniques that can be used.
The Technique dimension of the taxonomy differentiates between systems that change parameters, their structure or their context.
A system that changes its structure could, for example, be a datacenter, which starts new server instances on demand.
Robots are an example for systems that change their context.
This is achieved by, for example, moving the robot around.

\subparagraph*{Who}
After answering where, when, what and why changes should be performed, 
it is necessary to select who is responsible for these changes.
According to Salehie and Tahvildari it is also important to establish if the changes can be performed fully autonomous
or if the involvement of human operators is necessary.
The taxonomy does not directly address all of these concerns but states that:
"N/A (nature of a SAS leads to an automatic type of adaptation)" \cite{SurveyOnEngineeringApproaches}.

\subparagraph*{How}
Lastly, after determining the where, when, what, why and who, there needs to be a way
to perform the required changes. This is answered by asking how the changes should be performed
and corresponds to the Adaptation Control.
The three main aspects of the Adaptation Control are the degree of decentralization, the adaptation decision criteria
and the approach taken by the system.
The degree of decentralization ranges from systems which perform adaptations through a central component (fully centralized)
to systems where each component is responsible for its own adaptation (fully decentralized).
In between these two degrees are systems that employ a hybrid approach.
The adaptation decision criteria controls how the adaptation is achieved.
Krupitzer et al. propose the following possibilities:
\begin{itemize}[nosep]
    \item Models: The \acrshort{sas} updates e.g. domain models which results in behavioral changes
    \item Rules/Policies: The \acrshort{sas} changes its behavior based on rules or policies
    \item Goals: The \acrshort{sas} changes its behavior to meet some predefined set of goals
    \item Utility: The \acrshort{sas} changes its behavior to maximize or minimize a utility function
\end{itemize}
The approach divides \acrshort{sas} into those where the adaptation logic is part of the application logic
and those with separated adaptation and application logic.

\noindent After classifying \acrshort{sas} the following question can be asked: Which parts of a \acrshort{sas} can be optimized?
To answer this we will start by looking at which parts can not be optimized or do not benefit from optimization.

\noindent The first part that can not be optimized is the environment which provides the Reason dimension.
While the environment for a \acrshort{sas} can be chosen in a way which is most beneficial for the system
and can be influenced by actors,
the behavior of the systems environment can generally not be controlled.

\noindent Another dimension of \acrshort{sas} that can not be optimized, or is not useful to optimize,
is the Time dimension. This dimension is mostly a design decision on how the system should behave and be constructed.
It is also a question of how to handle uncertainty and the level of accepted risk.
A proactive system can prevent faults and degradation in Quality-of-Service metrics,
but it can also predict the wrong changes which can lead to a situation where the system itself generates faults by
reacting in a way that is contradictive to its goals.
A reactive system can not prevent faults like a proactive system,
but its behavior can be much more stable, because it only has to react to a change and not predict that change as well.

\noindent Lastly, the question of who is responsible can not be optimized, because the taxonomy simply answers it,
by referring to the name "\acrshort{sas}" which implies that the system itself is responsible for managing adaptations.

\noindent The three remaining dimensions can be optimized or can benefit from being adapted dynamically.
These are the Adaptation Control, the Level and the Technique.

\noindent The Approach and the Degree of Decentralization used by the Adaptation Control can not be optimized 
because they are design decisions of how the system is built.
However, the Adaptation Decision Criteria can be optimized. An optimization of the Adaptation Decision Criteria
could, for example, be to dynamically adapt the rules and policies at runtime to better reflect a changing environment.

\noindent Another dimension that can be optimized is the Technique. 
This can be optimized by changing what gets adapted by the system.

\noindent The last dimension that can be optimized is the Level, which can be done by dynamically changing at which level of the system
adaptations should be performed.
\newpage
\section{Proposal for classification of optimization approaches}
\label{ch:Proposal}

% TODO:
% - Deriving:
%     - SAS Taxonomy by Krupitzer et al
%     - How did Krupitzer get to his taxonomy?
%     - 5W+1H questions
%     - Classification proposed by: The Application of ML
% - Proposing:
%     - Explain each part:
%         - Purpose
%         - Time
%         - Technique
%         - Location
%         - Concern
%     - What to explain:
%         - Why?
%         - How is it different from the other properties?

% \begin{itemize}
%     \item Deriving a classification for optimization approaches for Self-Adaptive Systems.
%     \item Proposing a classification for optimization approaches for Self-Adaptive Systems.
% \end{itemize}

% \paragraph*{Literature for this section:} \begin{itemize}
%     \item "The Application of Machine Learning in Self-Adaptive Systems: A Systematic Literature Review" \cite{ApplicationOfMachineLearning}
% \end{itemize}

% TODO: Deriving a classification}
% TODO: reference for MAPE
% TODO: FUSION and other OA just update a knowledge base (MAPE-K) and dont directly adapt the system
% so maybe RAC is not the perfect name for that category: changing structure vs providing knowledge
% approach: not if RAC is internal/external but if the OA is internal/external
% This would also make the classification more relevant!
Using the idea from Weyns et al. 2012 paper FORMS \cite*{FORMS} to compose Self-Adaptive Systems from
layers of base and reflective components, 
an Optimization Approach for Self-Adaptive Systems can be thought of
as a reflective component in a layer above the Self-Adaptive System.
This allows the application of multiple Optimization Approaches and even the 
optimization of Optimization Approaches.
It also establishes a connection between Optimization Approaches and Self-Adaptive Systems
and leads to the realization that a classification for Optimization Approaches of Self-Adaptive Systems
should be based upon the same principles as the classifications for Self-Adaptive Systems.
Another benefit is a clear distinction between Self-Adaptive Systems and Optimization Approaches for Self-Adaptive Systems:
While Self-Adaptive Systems and Optimization Approaches are both composed of reflective components and not base components,
the Self-Adaptive System adapts base components, while the Optimization Approach adapts other reflective components.
\newline
\par


Following Krupitzer's et al, 2015 \cite*{SurveyOnEngineeringApproaches} taxonomy for Self-Adaptive Systems,
a classification for Optimization Approaches of Self-Adaptive Systems should answer 
the 5W+1H questions by Salehie and Tahvildari, 2009 \cite*{LandscapeAndResearchChallenges}.
These questions are: Where, When, What, Why, Who and How?
% These questions are:
% \begin{itemize}
%     \item Where is the need for change?
%     \item When should a change occur?
%     \item What should be changed?
%     \item Why should something be changed?
%     \item Who should change something?
%     \item How should something be changed?
% \end{itemize}

\begin{figure}[h!]
    \centering
    \includegraphics[width=0.6\columnwidth]{images/ClassificationProposal-OptimizationMAPEK.png}
    \caption{Mapping the optimization process to the MAPE-K feedback loop.}
    \label{fig:MappingOptMAPEK}
\end{figure}

Just like the adaptation process of Self-Adaptive Systems can be understood using
the MAPE-K (Monitor-Analyze-Plan-Execute with Knowledge) feedback loop by Kephart and Chess, 2003 \cite*{VisionOfAutonomicComputing}.
The process of optimizing a Self-Adaptive System can also be expressed using the MAPE-K feedback loop:
\begin{itemize}
    \item Firstly the optimization approach has to constantly monitor the Self-Adaptive System and its context.
    \item The data gathered from monitoring can then be analyzed to decided wether or not an optimization is necessary and what should be optimized.
    \item After deciding that something should be optimized, there needs to be a plan on how to optimize.
    \item Lastly the planned optimization can be executed.
    \item For all of this the optimization approach requires knowledge about the Self-Adaptive System and its context.
\end{itemize}

% TODO: Proposing a classification
% TODO: transition to proposal
\newpage
\begin{figure}
    \centering
    \includegraphics[width=0.8\columnwidth]{images/ClassificationProposal-WithDimensions.png}
    \caption{The proposed classification for Optimization Approaches for Self-Adaptive Systems}
    \label{fig:Proposal}
\end{figure}

\subparagraph*{Location}
Where in the system is an optimization necessary? \\
There are three main parts in a Self-Adaptive System that can require optimizations during its lifetime.
These are:
\begin{itemize}
    \item The Adaptation Control.
    \item The Level on which adaptations are performed
    \item The Technique that is used to perform adaptations.
\end{itemize}

\subparagraph*{Time}
When are optimizations performed? \\
Similarly to Self-Adaptive Systems, Optimization Approaches can be differentiated by comparing when they perform optimizations.
There are three different phases during the lifetime of a Self-Adaptive System where optimizations can occur.
These are:
\begin{itemize}
    \item At the runtime of the system, which can also be called the online phase.
    \item During the design time of the system.
    \item While training the system or during its offline phase.
\end{itemize}
The training phase can happen in parallel to the run and design time of the system.
Examples for this would be when the initial parameters for a Self-Adaptive System are chosen by training a domain model
or when generating a new model for the system by training an updated domain model parallel to the running system.

\subparagraph*{Technique}
What gets changed to perform the optimization? \\
What gets changed in a Self-Adaptive System relates to where an optimization is necessary.
Because of this, the following parts can be changed:
\begin{itemize}
    \item The Rules / Policies used by the Adaptation Decision Criteria.
    \item The Knowledge of the system, which for example can consist of domain models.
    \item The Adaptation Technique used by the Self-Adaptive System.
    \item The Level on which the Self-Adaptive System performs changes.
\end{itemize}
One could argue that the Goals or Utility functions of a Self-Adaptive System could also be changed to optimize its performance.
In most cases this is not advisable because Goals and Utility functions can encode information
like legal or safety parameters, of which the system has no intrinsic knowledge.

\subparagraph*{Purpose}
Why should an optimization occur? \\
This is perhaps the most important question for any Optimization Approach because it determines if an optimization is necessary.
There can be several reasons for performing an optimization in a Self-Adaptive System:
\begin{itemize}
    \item The system learned new information about for example its environment and needs to update its domain models.
    \item Because of external or internal changes, a Goal might not be satisfiable anymore.
    \item The difference of the actual versus the expected outcome of a rule / policy has exceeded a threshold.
    \item A Utility function needs to be maximized or minimized.
\end{itemize}

% TODO: Approach outside of Adaptation Control because of the original meaning
% of the question: level of automation vs human interaction.
% Self-Adaptive Systems: trivially should only be adapted by themselves.
% Optimization approaches: there might be human intervention necessary -> ML: learning trap, unpredicatability
\subparagraph*{Approach}
Who is responsible for performing the optimization? \\
\begin{itemize}
    \item Is there an internal component that performs changes or are they performed by an external actor.
    \item Which degree of decentralization is used?
    Is each component responsible for itself (fully decentralized), are all components optimized by a central entity (fully centralized)
    or is a hybrid approach used?
\end{itemize}

\subparagraph*{Reflective Adaptation Control}
How is the optimization applied to the system? \\
Just like the Self-Adaptive System can be described by its Adaptation Control,
which is responsible for applying changes to the system,
an optimization approach can also be described by how it applies its optimizations to the system.
In reference to FORMS by Weyns et al., 2012\cite*{FORMS}, which uses reflective operations to change
components in the system, the Adaptation Control of the optimization approach will be called Reflective Adaptation Control.

Unlike the Adaptation Control of Self-Adaptive Systems, 
the Reflective Adaptation Control does not have a dimension for the Approach, because it has its own separate dimension.
\begin{itemize}
    \item Adaptation Decision Criteria: \begin{itemize}
        \item Models: The system updates its models after receiving new information about its models.
        \item Rules / Policies: If the system detects, that it is in a state for which an adaptation rule/policy exists,
        it should perform that adaptation.
        \item Goals: Adaptation should be performed if the system does not meet pre-defined goals.
        \item Utility: An adaptation occurs to maximize or minimize a utility function.
    \end{itemize}
\end{itemize}

Because the proposed classification is based upon the MAPE-K feedback loop and the 5W+1H questions,
it is useful to map the classification to each of these models.
\begin{figure}[h!]
    \centering
    \begin{tabular}{|lcl|}
        \hline
        MAPE-K & & Optimization Approach \\
        \hline
        Monitor & & \\
        \hline
        Analyze & & Location\\
        \hline
        Plan & & Technique \\
        \hline
        Execute & & Reflective Adaptation Control \\
        \hline
    \end{tabular}
    \caption{How the MAPE-K feedback loop by Kephart and Chess, 2003\cite*{VisionOfAutonomicComputing}
    relates to the dimensions of the classification for Optimization Approaches for Self-Adaptive Systems.}
\end{figure}

\begin{figure}[h!]
    \centering
    \begin{tabular}{|lcl|}
        \hline
        5W+1H questions & & Optimization Approach \\
        \hline
        Where & & Location \\
        \hline
        When & & Time \\
        \hline
        What & & Technique \\
        \hline
        Why & & Purpose \\
        \hline
        Who & & Approach \\
        \hline
        How & & Reflective Adaptation Control \\
        \hline
    \end{tabular}
    \caption{How the 5W+1H questions by Salehie and Tahvildari, 2009 \cite*{LandscapeAndResearchChallenges}
    relate to the dimensions of the classification for Optimization Approaches for Self-Adaptive Systems.}
    \label{fig:5W1HProposal}
\end{figure}
\newpage
\section{Classifying existing optimization approaches}
\label{ch:Existing}

% TODO:
% - Write a short explanation of each approach:
%     - FUSION
%     - Multi-Agent system
%     - Learning revised models
%     - Cloud performance and security
%     - FIoT
% - Classify each approach:
%     - FUSION
%     - Multi-Agent system
%     - Learning revised models
%     - Cloud performance and security
%     - FIoT

% \begin{itemize}
%     \item Classifying a selection of existing optimization approaches using the proposed classification.
% \end{itemize}

% \paragraph*{Literature for this section:} \begin{itemize}
%     \item "FUSION: a framework for engineering self-tuning self-adaptive software systems" \cite{FUSION}
%     \item "A multi-agent systems approach to autonomic computing" \cite{MultiAgentSystem}
%     \item "Learning revised models for planning in adaptive systems" \cite{LearningRevisedModels}
%     \item "Using a multi-agent system and artificial intelligence for monitoring and improving the cloud performance and security" \cite{ImprovingCloudPerformanceAndSecurity}
%     \item "FIoT: An agent-based framework for self-adaptive and self-organizing applications based on the Internet of Things" \cite{FIoT}
%     \item FESAS by Krupitzer? TODO: research
% \end{itemize}

\subparagraph*{FUSION}
\begin{itemize}
    \item Location:
    \item Time:
    \item Technique:
    \item Purpose:
    \item Approach:
    \item Reflective Adaptation Control:
\end{itemize}
\newpage
\section{Conclusion}
\label{ch:Conclusion}

\begin{itemize}
    \item Recommending future research directions: \begin{itemize}
        \item Applying the proposed classification to more existing optimization approaches.
        \item Possible directions for new optimization approaches.
    \end{itemize}
    \item What are the limitations of this paper?
\end{itemize}

%% --------------------
%% |   Bibliography   |
%% --------------------

%% Add entry to the table of contents for the bibliography
\newpage
\printbibliography[heading=bibintoc]

\end{document}